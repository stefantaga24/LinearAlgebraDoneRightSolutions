\documentclass[11pt]{article}

\usepackage{amsmath, amssymb, amsthm}
\usepackage[margin=1in]{geometry}
\usepackage{enumitem}

% Theorem-style environments
\theoremstyle{definition}
\newtheorem{exercise}{Exercise}

\theoremstyle{remark}
\newtheorem*{solution}{Solution}

\title{Linear Algebra Done Right -- Chapter 3 Solutions}
\author{}
\date{}

\begin{document}

\maketitle

\section{Exercises 3B}

\begin{exercise}
    Give an example of a linear map $T$ widh dim null $T$ = 3 and dim range $T$ = 2.
\end{exercise}

The intuition behind the solution is that we can define a linear map just on the values 
on a basis of the linear space V we choose. I will choose V to be $\mathbb{R}^5$ and 
$T \in \mathcal{L(V)}$.

Therefore, I define $T:\mathbb{R}^5 \rightarrow \mathbb{R}^5$ with: 
\begin{itemize}
    \item{T ((1,0,0,0,0))=0}
    \item{T ((0,1,0,0,0))=0}
    \item{T ((0,0,1,0,0))=0}
    \item{T ((0,0,0,1,0))= (0,0,0,1,0)}
    \item{T ((0,0,0,0,1))= (0,0,0,0,1)}
\end{itemize}

It is clear that null $T$ has a basis the first 3 basis vectors of $\mathbb{R}^5$ and 
the range of $T$ has as a basis the last 2 basis vectors of $\mathbb{R}^5$.

\begin{exercise}
    Suppose $S,T \in \mathcal{L}(V)$ such that range $S$ $\subseteq$ null $T$. Prove that $(ST)^2$=0. 
\end{exercise}

\begin{solution}
$(ST)^2(v)=(ST)(ST(v))$. We have that $ST(V) \in$ range $S$. Therefore $ST(V) \in$ null $T$.

Which means that $T(ST(v))=0,\forall v \in V$. Therefore $(ST) ^2(v)=S(0)=0$, because $S$ is a linear map.
\end{solution}

\begin{exercise}
    Suppose $v_1,\cdots ,v_m$ is a list of vectors in $V$. Define $T \in (\mathcal{F}^{m},V)$ by: 
    \[
    T(z_1,\cdots,z_m)=z_1v_1+\cdots+z_mv_m.
    \]
\begin{itemize}
    \item What property of $T$ corresponds to $v_1,\dots,v_m$ spanning $V$?
    \item What property of $T$ corresponds to the list $v_1,\dots,v_m$ being linearly independent?
\end{itemize}
\end{exercise}
\begin{solution} To answer the questions in order:
    \begin{itemize}
    \item If $v_1,\dots,v_m$ span V, then $T$ would be surjective as practically range $T$ is just span $V$.

    \item If $v_1,\dots,v_m$ are linearly independent, this means that $T(z_1,\dots,z_m)=0$ only if $z_1=\cdots=z_m=0$, therefore dim null $T$ =0 , so it is injective.
    \end{itemize}
\end{solution}

\end{document}