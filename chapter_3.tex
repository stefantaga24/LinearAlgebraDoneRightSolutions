\documentclass[11pt]{article}

\usepackage{amsmath, amssymb, amsthm}
\usepackage[margin=1in]{geometry}
\usepackage{enumitem}

% Theorem-style environments
\theoremstyle{definition}
\newtheorem{exercise}{Exercise}

\theoremstyle{remark}
\newtheorem*{solution}{Solution}

\title{Linear Algebra Done Right -- Chapter 3 Solutions}
\author{}
\date{}

\begin{document}

\maketitle

\section{Exercises 3B}

\begin{exercise}
    Give an example of a linear map $T$ widh dim null $T$ = 3 and dim range $T$ = 2.
\end{exercise}

The intuition behind the solution is that we can define a linear map just on the values 
on a basis of the linear space V we choose. I will choose V to be $\mathbb{R}^5$ and 
$T \in \mathcal{L(V)}$.

Therefore, I define $T:\mathbb{R}^5 \rightarrow \mathbb{R}^5$ with: 
\begin{itemize}
    \item{T ((1,0,0,0,0))=0}
    \item{T ((0,1,0,0,0))=0}
    \item{T ((0,0,1,0,0))=0}
    \item{T ((0,0,0,1,0))= (0,0,0,1,0)}
    \item{T ((0,0,0,0,1))= (0,0,0,0,1)}
\end{itemize}

It is clear that null $T$ has a basis the first 3 basis vectors of $\mathbb{R}^5$ and 
the range of $T$ has as a basis the last 2 basis vectors of $\mathbb{R}^5$.

\begin{exercise}
    Suppose $S,T \in \mathcal{L}(V)$ such that range $S$ $\subseteq$ null $T$. Prove that $(ST)^2$=0. 
\end{exercise}

\begin{solution}
$(ST)^2(v)=(ST)(ST(v))$. We have that $ST(V) \in$ range $S$. Therefore $ST(V) \in$ null $T$.

Which means that $T(ST(v))=0,\forall v \in V$. Therefore $(ST) ^2(v)=S(0)=0$, because $S$ is a linear map.
\end{solution}

\begin{exercise}
    Suppose $v_1,\cdots ,v_m$ is a list of vectors in $V$. Define $T \in (\mathcal{F}^{m},V)$ by: 
    \[
    T(z_1,\cdots,z_m)=z_1v_1+\cdots+z_mv_m.
    \]
\begin{itemize}
    \item What property of $T$ corresponds to $v_1,\dots,v_m$ spanning $V$?
    \item What property of $T$ corresponds to the list $v_1,\dots,v_m$ being linearly independent?
\end{itemize}
\end{exercise}
\begin{solution} To answer the questions in order:
    \begin{itemize}
    \item If $v_1,\dots,v_m$ span V, then $T$ would be surjective as practically range $T$ is just span $V$.

    \item If $v_1,\dots,v_m$ are linearly independent, this means that $T(z_1,\dots,z_m)=0$ only if $z_1=\cdots=z_m=0$, therefore dim null $T$ =0 , so it is injective.
    \end{itemize}
\end{solution}

\begin{exercise}
    Show that X = $\{$ $T \in \mathcal{L}(\mathbb{R}^5,\mathbb{R}^4)$: dim null $T > 2$ $\}$ is not a subspace of $\mathcal{L}(\mathbb{R}^5,\mathbb{R}^4)$
\end{exercise}

\begin{solution} My intuition for this exercise was just to build two linear maps that had no overlap of values over the basis. So in one case,
one of the maps would map a basis vector to 0 and in another it would map the basis vector to itself.

Therefore, let $v_1,\dots,v_5$ be a basis of $\mathbb{R^5}$. I define the following maps $T_1$ and $T_2$:
\begin{itemize}
    \item $T_1(v_1)=v_1$,$T_1(v_2)=v_2$,$T_1(v_3)=0$,$T_1(v_4)=0$,$T_1(v_5)=0$
    \item $T_2(v_1)=0$,$T_2(v_2)=0$,$T_2(v_3)=v_3$,$T_2(v_4)=v_4$,$T_2(v_5)=0$
\end{itemize}

Both of them clearly have dim null $T_1$ = dim null $T_2$ = 3. 

However, when we calculate $T_1+T_2$ we get: 
\begin{itemize}
    \item $(T_1+T_2)(v_1)=v_1$, $(T_1+T_2)(v_2)=v_2$, $(T_1+T_2)(v_3)=v_3$, $(T_1+T_2)(v_4)=v_4$, $(T_1+T_2)(v_5)=v_5$
\end{itemize}

Therefore, dim null $T_1+T_2$ = 1 , which means that $T_1+T_2 \notin X$ which shows
that the set X is not closed under addition, therefore it cannot be a subspace. 
\end{solution}

\begin{exercise}
Give an example of $T \in \mathcal{L}(\mathbb{R}^4)$ such that range $T$ = null $T$. 
\end{exercise}
\begin{solution}
My intuition behind this problem was that if range $T$ = null $T$, then T(T(x))=0, $\forall{x} \in \mathbb{R}^4$. 
Now, I did "cheat" a little bit by thinking that $T$ was a matrix and just looking for a matrix with $T^2=0$ , where 0 here is the null $4 \times 4$ matrix.

Therefore, the final map that I have reached is: $T(x_1,x_2,x_3,x_4)=(x_1-x_4,x_2-x_3,x_2-x_3,x_1-x_4)=(x_1-x_4)*(1,0,0,1)+(x_2-x_3)*(0,1,1,0)$. It is clear that
the range is a linear combination between $(1,0,0,1)$ and $(0,1,1,0)$ which are linearly independent, therefore dim range $T$ = 0. We have that $$T(T(x))=0$$ which means
that range $ T \subseteq $null $T$. 

Now I need to prove that null $T \subseteq$ range $T$.

Let $y \in $ null $T$ , be an arbitrary vector in $null T$. Therefore, $T(y_1,y_2,y_3,y_4)=0$ which means that $y_1=y_4,y_2=y_3$.

I will choose $x_1=y_1,x_2=y_2,x_3=0,x_4=0$ , and get that $T(x_1,x_2,x_3,x_4)=(y_1,y_2,y_2,y_1)=(y_1,y_2,y_3,y_4)$. Therefore I have 
found $x_1,x_2,x_3,x_4$ such that $T(x_1,x_2,x_3,x_4)=(y_1,y_2,y_3,y_4)$ so $y \in $ range $T$. Therefore null $T$ $\subseteq$ range $T$.

So null $T$ = range $T$.
\end{solution}

\begin{exercise}
    Prove that there does not exist $T \in \mathcal{L}(\mathbb{R}^5)$ such that range $T$ = null $T$.
\end{exercise}
\begin{solution}
    We know that dim range $T$ + dim null $T$ = dim $\mathbb{R}^5$ , which means 2*dim range $T$ = 5 which is not possible as dim range $T$ is an integer.
\end{solution}

\begin{exercise}
    Suppose $V$ and $W$ are finite-dimensional with 2 $\leq$ dim V $\leq$ dim W. Show that $\{T \in \mathcal{L}(V,W): T \text{ is not injective}\}$ is not a subspace of $\mathcal{L}(V,W)$.
\end{exercise}
\begin{solution}
    I will build two maps that are not injective, but whose sum is. I will denote dim $V$ = n, dim $W$ = m.
    Let $v_1,\dots,v_n$ be a basis of $V$ and $w_1,\dots,w_m$ be a basis of $W$. 
    I can create the following mappings, using the fact that $2 \leq n \leq m$:
    \begin{itemize}
        \item $T_1(v_1)=w_1, T_1(v_2)=w_1, T_1(v_i)=w_i, \forall i \in \{3,\dots,n\}$
        \item $T_2(v_1)=w_2, T_2(v_2)=-w_2, T_2(v_i)=w_i, \forall i \in \{3,\dots,n\}$
    \end{itemize}
    They are both not injective as their nullspaces do not have dimension $0$, clearly $T_1(v_1-v_2)=0$ and $T_2(v_1+v_2)=0$. 

    If we look at $T_1+T_2$, we find that $(T_1+T_2)(v_1)=w_1+w_2,(T_1+T_2)(v_2)=w_1-w_2,\dots,(T_1+T_2)(v_i)=2*v_i, \forall i \in \{3,\dots,n\}$.
    Let $x \in $ null $(T_1+T_2)$. We have that $(T_1+T_2)(x)=0$, we can write $x$ uniquely according to the basis $v_1,\dots,v_n$ and get that: 
    $(T_1+T_2)(\alpha_1 v_1+\cdots+\alpha_n v_n)=0$. Therefore $\alpha_1(w_1+w_2)+\alpha_2(w_1-w_2)+\sum_{i=3}^n 2*\alpha_i w_i=0$.

    So we get that $(\alpha_1+\alpha_2)w_1 + (\alpha_1-\alpha_2)w_2+\sum_{i=3}^{n} w_i =0$. We know that the w's are linearly independent, 
    therefore $\alpha_i=0, \forall i \in \{3,\dots,n\}$ and $\alpha_1+\alpha_2=0$ and $\alpha_1-\alpha_2=0$. This means that $\alpha_1=\alpha_2=0$, so
    x = 0. Therefore, null $T_1+T_2=\{0\}$, so it is injective, therefore the set given in the hypothesis is not a subspace.


\end{solution}

\end{document}